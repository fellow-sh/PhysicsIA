\section*{Conclusion}

With regards to the hypothesis that current is linearly proportional to the magnetic force exerted on a wire, the methods and experiment used ultimately resulted in data that partially supports the hypothesis.
Having not considered that the strength of the magnetic field varies with position, there exists systematic error which makes it difficult to judge whether the relation between the current and force is linear.
In addition, the design of the experiment had created some complications with measurement and resulted in minor skewing of the data.
Nevertheless, from Figure \ref{fig:procplot}, it is clear that they are correlated such that they are proportional.
This evidence is consistent with Lorentz force, which describes the connection between electric charge and magnetism.

\section*{Improvements for Future Investigation}

As suggested in the evaluation, conducting the experiment at a larger scale would make angle measurements easier, potentially reduce current fluctuations, and resolve the issue of the wire's proximity to the magnets.
Using copper tape and a copper pipe will further reduce current fluctuations and issues regarding friction.
Stronger magnets with larger surface area, preferably neodymium bar magnets with the poles on the long edge, as well as utilizing larger currents can allow for a wider range of deflection angles.
And finally, to prevent the same mistakes of this investigation, the magnet composition and grade---specifically its magnetic flux density---should be known, and variation of field strength around the two magnets at a specific separation distance should be measured prior to beginning the trials.