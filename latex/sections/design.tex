\section*{Experiment Design}

\subsection*{Hypothesis}

As the current going through the wire increase, the force acting on the wire due to a magnetic field will linearly increase. Hence, the deflection angle of the pendulum apparatus will also increase.

\subsection*{Variables}

\textbf{Independent variable} --- current flowing through the wire of the pendulum, achieved through varying output voltage from the power supply.
Due to the inconsistent resistance of the circuit, caused by changing contact of the wire with the supports as it swings, exact current values cannot be achieved.
The target current values were $0.20\si{\ampere}$, $0.40\si{\ampere}$, $0.60\si{\ampere}$, $0.80\si{\ampere}$, $1.00\si{\ampere}$, $1.20\si{\ampere}$, and $1.40\si{\ampere}$.
These variables were measured with an uncertainty of $\pm0.01$ based off the readings from the power supply.

\textbf{Dependent variable} --- angle of deflection of the pendulum. The angle was measured through by imaging the deflecting pendulum at a flat plane and constant distance, then using a measurement tool from GIMP, an image processing software. The measurement uncertainty was $\pm0.01$.

\textbf{Controlled variables} --- the mass of the wire; $4.9\pm0.1\si{\gram}$ --- length of the copper wire under the magnetic field; $4.61\pm0.05\si{\centi\meter}$ --- pendulum length; $5.56\pm0.05\si{\centi\meter}$ --- distance between magnets; $3.12\pm0.05\si{\centi\meter}$.
The pendulum is kept perpendicular to the magnetic field lines between the magnets.

\subsection*{Procedure}

\begin{enumerate}
	\item Lay out a breadboard 
\end{enumerate}
