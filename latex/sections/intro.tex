\section*{Introduction}

As the world transitions from internal combustion engines to electric motors, I have begun to wonder how exactly electricity is used in vehicles.
More specifically, how electrical force is transformed into physical force.
Despite what I have learned from the school curriculum, there is not significant amount material covering electricity to begin with.
Having worked with DC motors for robotics projects when I was younger, all I understood was that electricity made these motors spin.
Later on, through IB Physics, I know that electric current in a wire can create a magnetic field and can therefore interact with other magnets, but this is described in detail.
After some research, it seems that Lorentz force is at the heart of how electric motors operate.\footcite{kramer2}
There are a number of homemade experiments on the internet that demonstrate Lorentz force, particularly by using a pendulum,
but only a small portion of them can quantify it.
In hopes of better understanding this force, I have asked the following research question:

\textbf{How does varying electric current affect the force exerted on a wire in a magnetic field?}
